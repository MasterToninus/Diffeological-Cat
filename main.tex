

\documentclass[a4paper,11pt,fleqn]{article}  %Per La Stampa
\usepackage[a4paper, margin=2cm]{geometry}

\usepackage{amsmath, amssymb, hyperref}

%\usepackage[subpreambles=true]{standalone}
\usepackage{commath}

\usepackage[english]{babel}
\usepackage[utf8]{inputenc}


\usepackage{graphicx}


\usepackage[basic,cat]{./Math-Symbols-List/toninus-math-symbols}
\usepackage{./Latex-Theorem/theoremtemplate}
\usepackage{./visualcat}


%Diagrammatic
\usepackage{tikz-cd}
\usepackage{xcolor}
\usepackage{fdsymbol}




\title{Reading Talk \\ Categorical approach to diffeological spaces}
%\title{On the Convenient Category of Diffeological Spaces}
\author{Tony}
\date{\today}

\begin{document}

\maketitle

\begin{abstract}
Diffeological spaces generalize smooth manifolds, providing a robust framework for studying "generalized spaces." A diffeological space consists of a set $X$ equipped with a collection of "plots" — smooth maps from open subsets of Euclidean spaces into $X$ — that satisfy three basic axioms. While individual diffeological spaces may exhibit geometric pathologies absent in smooth manifolds, the category $\mathbf{Diff}$ of all diffeological spaces possesses numerous desirable categorical properties that the category of smooth manifolds lacks.

This note will provide a concise introduction to diffeological spaces, the concept of a concrete site, and concrete sheaves. We will demonstrate that $\mathbf{Diff}$ can be identified as a category of "concrete sheaves on a concrete site," making it an archetypal example of a "generalized space." The remainder of the discussion will focus on the structural properties of this class of categories. Notably, any category of concrete sheaves on a concrete site — including $\mathbf{Diff}$ — is a quasitopos, enjoying the existence of all limits and colimits.
\end{abstract}


The foundational observation for a categorical approach to diffeological spaces is that a \emph{diffeology} \(\mathcal{D}\) on a set \(X\) can be understood as a \emph{presheaf} on the category of open subsets of Euclidean spaces, with smooth maps as morphisms. Specifically, to each open set \(U \subseteq \mathbb{R}^n\), it assigns a subset of \(\text{Set}(U, X)\), the set of all functions from \(U\) to \(X\) prescribed to be "smooth". 
A \emph{diffeological space} is then defined as a set equipped with a diffeology.





\section{Prerequisites}

\subsection{Presheaves}
Let $\cat$ be an arbitrary category\footnote{All categories considered are locally small.}.

\begin{definition}[Presheaf over $\cat$ \cite{nlab:presheaf}]
A preshef is a functor 
$ F: \cat^\op \to \Set$.
\\
The collection of all presheaves forms the category $[\cat^\op,\Set]$ whose objects are presheaves and morphisms are natural transformations.
\end{definition}

\begin{example}[Yoneda Embedding\cite{nlab:yoneda_embedding}]
For $\cat$ a locally small category, every object $c\in \cat$ induces a presheaf $Y(c)$ called \emph{the representable presheaf represented by $c$}.
This assignment extends to a functor called \emph{Yoneda embedding}

\begin{equation}
	.
\end{equation}

Hence $Y$ sends any object $c\in\cat$ to the the \emph{representable presheaf}\cite{nlab:representable_functor} which assigns to any other object $d \in \cat$ the hom-set of morphisms from $d$ into $c$.
The \emph{Yoneda lemma}\cite{nlab:yoneda_lemma} implies that this functor is full and faithful and hence realizes $\cat$ as a full subcategory inside its category of presheaves.
 
\end{example}

\begin{example}\label{ex:topolo-presheaf}
Let $X$ be a topological space, one can introduce the category of open sets:
%
	\catdef{\cat{Op}(X)}{todo}
	{todo}
%
Where objects are elements of the topology of $X$ and morphisms are inclusion of open set into another one.
A presheaf $\cat[Op](X)^\op \to \Set$ is exactly what we call a presheaf in topology.
\end{example}
In topology, a sheaf is defined as a presheaf that satisfies a specific gluing condition. This condition can be expressed using categorical language through the concept of sites.



\subsection{Sites}
Let $\cat$ be any small category.

\begin{definition}[Site]
	A site is a small category equipped with a coverage.
\end{definition}

\begin{definition}[Coverage]
	A Coverage is the assigment to each object $U \in \cat$ of a collection of families of arrows $\{f_i:U_i\to U\}_{i\in I}$, called \emph{covering families}, such that
\begin{enumerate}
	\item For any given $\{f_i:U_i\to U\}_{i\in I}$ covering family and $g:V\to U$ morphism, there exists a covering family $\{h_j:V_j\to V\}$ such that each composite $g \circ h_j$ factors through some $f_i$.
	\begin{displaymath}
		\begin{tikzcd}
		V_j \ar[r,"k"] \ar[d,"h_j"] & U_i \ar[d,"f_i"] \\
		V \ar[r,"g"] & W
		\end{tikzcd}
	\end{displaymath}
	%
	(NB. The logic here is: $\forall$ $f$ covering family , $\forall g$ morphism, $\exists h$ covering family such that $ \forall j, \exists i, \exists k \colon$ such that the above commutes.)
\end{enumerate}
\end{definition}

\begin{example}
	Given a topological space $X$, the category $\cat{Op}(X)$ is a site.
	Here a covering family for an open set $U$ is a family of open sets $U_i \hookrightarrow U$ such that $U_{i\in I} = U$. 
\end{example}
Actually it is more! Is a Grothendieck pretopology:

\begin{definition}[Grothendieck pretopology]
A Grothendieck pretopology on $\cat$ is an assignment to each object  $U$ of $C$ of a collection of families $\{U_i \to U\}$ of morphisms, called \emph{covering families} such that
\begin{enumerate}
	\item (Stability under base changes.)
The collection of covering families is stable under pullback: if $\{U_i \to U\}$ is a covering family and $f : V \to U$ is any morphism in $C$, then $\{f^* U_i \to V\}$ exists and is a covering family;
	\item (Stability under composition.)
If $\{U_i \to U\}_{i \in I}$ is a covering family and for each $i$ also $\{U_{i,j} \to U_i\}_{j \in J_i}$ is a covering family, then also the family of composites $\{U_{i,j} \to U_i \to U\}_{i\in I, j \in J_i}$ is a covering family.
	\item (Isomorphisms cover.)
Every family consisting of a single isomorphism $\{V \stackrel{\cong}{\to}U\}$ is a covering family;
\end{enumerate}
\end{definition}

N.b. condition 1) of the previous two definitions are equivalent.

A \emph{concrete site} is a site whose objects can be thought of as sets with extra structure. 
\begin{definition}[Concrete site]
	A \emph{concrete site} is a site $\cat$ with a terminal object $*$ such that
	\begin{enumerate}
		\item the functor $\Hom_{\cat}(*,-) : \cat \to \Set$ is a faithful functor;
		\item for every covering family $\{f_i : U_i \to U\}$ in $\cat$ the morphism 

   $$
     \coprod_i \Hom_{\cat} (*,f_i) : \coprod_i \Hom_{\cat} (*, U_i) \to \Hom_{\cat} (*, U) 
   $$
   is surjective.
	\end{enumerate}
\end{definition}


In a category of presheaves on a concrete site one can consider concrete presheaves.

\begin{definition}[Concrete presheaf]
 TRICKY!!! guarda note vecchie!
\end{definition}



\subsection{Sheaves}

\begin{definition}[Sheaf]
Let $(\cat,J)$ be a site in the form of a small category $\cat$ equipped with a coverage $J$.
\\
A presheaf $A \in [\cat^\op,\Set]$ is a \emph{sheaf} with respect to $J$ if
\begin{itemize}
	\item for every covering family $\{p_i : U_i \to U\}_{i \in I}$ in $J$;
	\item and for every compatible family of elements, given by tuples $(s_i \in A(U_i))_{i \in I}$ such that for all $j,k \in I$ and all morphisms $U_j \stackrel{f}{\leftarrow} K \stackrel{g}{\to} U_k$ in $C$ with $p_j \circ f = p_k \circ g$ we have $A(f)(s_j) = A(g)(s_k) \in A(K)$
\end{itemize}
then
\begin{itemize}
	\item there is a  unique  element $s \in A(U)$ such that $A(p_i)(s) = s_i$ for all $i \in I$.
\end{itemize}
\end{definition}

\begin{definition}[Subcanonical sites]
 We call a site $\cat$ subcanonical if every representable functor $Y(c): \cat^\op \to \Set$ is a sheaf.
\end{definition}

The category of all \href{https://ncatlab.org/nlab/show/sheaf#GeneralDefinitionAbstractly}{sheaves} on a site is extremely nice: it is a \href{https://ncatlab.org/nlab/show/topos#definitions}{topos}. 
The category of concrete sheaves on a concrete site is also nice, but slightly less so: it is a ‘quasitopos’. 

\begin{definition}[{Quasitopos \cite[Def. 3.1]{Baez2011}}]
A quasitopos is a \href{https://ncatlab.org/nlab/show/locally+cartesian+closed+category#definition}{locally Cartesian closed} category with finite colimits and a \href{https://ncatlab.org/nlab/show/subobject+classifier#weak_subobject_classifier}{weak subobject classifier}.
\end{definition}

\begin{theorem}[{\cite[Thm. 5.25]{Baez2011}}]
For any concrete site $\cat[D]$, the category of concrete sheaves over $\cat[D]$ is a quasitops with all (small) limits an colimits.
\end{theorem}




\section{ Diffeological Spaces as a Category of Concrete Sheaves}
\begin{itemize}
    \item The site of Euclidean spaces $\textbf{Eucl}$
    \item $\textbf{Eucl}$ is subcanonical
    \item $\textbf{Eucl}$ is concrete
    \item Sheaves on $\textbf{Eucl}$ (Definition 2.1.5)
    \item Concrete presheaves on $\textbf{Eucl}$ (Definition 2.1.9)
    \item \textbf{Theorem:} 2.1.11: The category of diffeological spaces is equivalent to concrete sheaves over $\textbf{Eucl}$
\end{itemize}

\section{Some categorical Properties of $\textbf{Diff}$}
\subsection*{Diffeological Mapping Spaces}
\begin{itemize}
    \item $\textbf{Diff}$ is \href{https://ncatlab.org/nlab/show/locally+cartesian+closed+category}{locally Cartesian closed}
    \item Slices have all finite products and exponential objects
    \item Exponential objects in $\textbf{Diff}$
\end{itemize}

\subsection*{Discrete and Indiscrete Diffeology}
\begin{itemize}
    \item View these as the left and right adjoints of the forgetful functor
\end{itemize}

\subsection*{Class of Morphisms}
\begin{itemize}
    \item Monomorphisms and epimorphisms
\end{itemize}

\subsection*{Limits}
\begin{itemize}
    \item Computing limits and colimits
\end{itemize}



\appendix

\section*{Appendix: Basic stuff}

\begin{definition}[Concrete category]\label{def:concrete_cat}
A category $\cat$ is called \emph{concrete} if it is equipped with a faithful functor 
$$
|\cdot| : \cat \to \Set.
$$
This functor, referred to as the \emph{forgetful functor}, assigns to each object $c \in \cat$ its \emph{underlying set} \(|c|\).
\end{definition}

\begin{remark}\label{rem:concrete_terminal}
	In practice, the objects of a concrete category are sets equipped with additional structure, and the morphisms are structure-preserving maps between these sets.\\
	If $\cat$  has a terminal object $*$, the concrete structure is often determined by the \emph{functor of points}, given by $(c \mapsto \Hom_{\cat} (*, c))$.

\end{remark}

%=========================================================================#
% Bibliography (BibTex)
% https://arxiv.org/hypertex/bibstyles/
%=========================================================================#
			\bibliographystyle{hep}
			\bibliography{bibliography}


\end{document}
