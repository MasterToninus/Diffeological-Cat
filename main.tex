

\documentclass[a4paper,11pt,fleqn]{article}  %Per La Stampa
\usepackage[a4paper, margin=2cm]{geometry}

\usepackage{amsmath, amssymb, hyperref}

\usepackage[subpreambles=true]{standalone}
\usepackage{commath}

\usepackage[english]{babel}
\usepackage[utf8]{inputenc}


\usepackage{graphicx}
\usepackage{hyperref}

\usepackage[basic,cat]{./Math-Symbols-List/toninus-math-symbols}
\usepackage{./Latex-Theorem/theoremtemplate}



%Diagrammatic
\usepackage{tikz-cd}
\usepackage{xcolor}
\usepackage{fdsymbol}
	% bulk of arrows and objects
	\newcommand{\bulk}{\blacklozenge}
	% such that
	\newcommand{\St}{\textrm{s.t.}}
	% id est
	\newcommand{\ie}{\textrm{\emph{i.e.}}}
	% commuting
	\newcommand{\commute}{\circledequal}%\circlearrowleft




\title{Reading Talk \\ Categorical approach to diffeological spaces}
%\title{On the Convenient Category of Diffeological Spaces}
\author{Tony}
\date{\today}

\begin{document}

\maketitle

\begin{abstract}
Diffeological spaces generalize smooth manifolds, providing a robust framework for studying "generalized spaces." A diffeological space consists of a set $X$ equipped with a collection of "plots"—smooth maps from open subsets of Euclidean spaces into $X$—that satisfy three basic axioms. While individual diffeological spaces may exhibit geometric pathologies absent in smooth manifolds, the category $\mathbf{Diff}$ of all diffeological spaces possesses numerous desirable categorical properties that the category of smooth manifolds lacks.

This talk will provide a concise introduction to diffeological spaces, the concept of a concrete site, and concrete sheaves. We will demonstrate that $\mathbf{Diff}$ can be identified as a category of "concrete sheaves on a concrete site," making it an archetypal example of a "generalized space." The remainder of the discussion will focus on the structural properties of this class of categories. Notably, any category of concrete sheaves on a concrete site—including $\mathbf{Diff}$—is a quasitopos, enjoying the existence of all limits and colimits.
\end{abstract}

\section{Prerequisites}

\subsection{Presheaves}
\begin{itemize}
    \item \textbf{Definition:} \href{https://ncatlab.org/nlab/show/presheaf}{Presheaf}
    \item \href{https://ncatlab.org/nlab/show/Yoneda+embedding}{Yoneda Embedding}
    \item \textbf{Definition:} \href{https://ncatlab.org/nlab/show/representable+functor}{Representable Presheaf}
    \item Concrete categories and presheaves
\end{itemize}

\subsection{Sites}
\begin{itemize}
    \item \textbf{Definition:} Site
    \item \textbf{Definition:} Coverage
    \item \textbf{Definition:} Grothendieck Pretopology
    \item \textbf{Definition:} Sieves
    \item \textbf{Definition:} \href{https://ncatlab.org/nlab/show/Grothendieck+topology#original_definition}{Grothendieck Topology}
    \item \textbf{Definition:} Concrete sites
    \item \textbf{Definition:} Subcanonical sites
\end{itemize}

\subsection{Sheaves}
\begin{itemize}
    \item \textbf{Definition:} \href{https://ncatlab.org/nlab/show/sheaf#GeneralDefinitionAbstractly}{Sheaf}
    \item \textbf{Definition:} \href{https://ncatlab.org/nlab/show/Grothendieck+topos}{Grothendieck Topos}
    \item \textbf{Definition:} Concrete Sheaves
    \item \textbf{Theorem:} 5.25 (BH11): The category of concrete sheaves is a quasitopos
\end{itemize}

\section{ Diffeological Spaces as a Category of Concrete Sheaves}
\begin{itemize}
    \item The site of Euclidean spaces $\textbf{Eucl}$
    \item $\textbf{Eucl}$ is subcanonical
    \item $\textbf{Eucl}$ is concrete
    \item Sheaves on $\textbf{Eucl}$ (Definition 2.1.5)
    \item Concrete presheaves on $\textbf{Eucl}$ (Definition 2.1.9)
    \item \textbf{Theorem:} 2.1.11: The category of diffeological spaces is equivalent to concrete sheaves over $\textbf{Eucl}$
\end{itemize}

\section{Categorical Properties of $\textbf{Diff}$}
\subsection*{Diffeological Mapping Spaces}
\begin{itemize}
    \item $\textbf{Diff}$ is \href{https://ncatlab.org/nlab/show/locally+cartesian+closed+category}{locally Cartesian closed}
    \item Slices have all finite products and exponential objects
    \item Exponential objects in $\textbf{Diff}$
\end{itemize}

\subsection*{Discrete and Indiscrete Diffeology}
\begin{itemize}
    \item View these as the left and right adjoints of the forgetful functor
\end{itemize}

\subsection*{Class of Morphisms}
\begin{itemize}
    \item Monomorphisms and epimorphisms
\end{itemize}

\subsection*{Limits}
\begin{itemize}
    \item Computing limits and colimits
\end{itemize}

\end{document}
